% Options for packages loaded elsewhere
\PassOptionsToPackage{unicode}{hyperref}
\PassOptionsToPackage{hyphens}{url}
%
\documentclass[
]{article}
\usepackage{amsmath,amssymb}
\usepackage{iftex}
\ifPDFTeX
  \usepackage[T1]{fontenc}
  \usepackage[utf8]{inputenc}
  \usepackage{textcomp} % provide euro and other symbols
\else % if luatex or xetex
  \usepackage{unicode-math} % this also loads fontspec
  \defaultfontfeatures{Scale=MatchLowercase}
  \defaultfontfeatures[\rmfamily]{Ligatures=TeX,Scale=1}
\fi
\usepackage{lmodern}
\ifPDFTeX\else
  % xetex/luatex font selection
\fi
% Use upquote if available, for straight quotes in verbatim environments
\IfFileExists{upquote.sty}{\usepackage{upquote}}{}
\IfFileExists{microtype.sty}{% use microtype if available
  \usepackage[]{microtype}
  \UseMicrotypeSet[protrusion]{basicmath} % disable protrusion for tt fonts
}{}
\makeatletter
\@ifundefined{KOMAClassName}{% if non-KOMA class
  \IfFileExists{parskip.sty}{%
    \usepackage{parskip}
  }{% else
    \setlength{\parindent}{0pt}
    \setlength{\parskip}{6pt plus 2pt minus 1pt}}
}{% if KOMA class
  \KOMAoptions{parskip=half}}
\makeatother
\usepackage{xcolor}
\usepackage[margin=1in]{geometry}
\usepackage{color}
\usepackage{fancyvrb}
\newcommand{\VerbBar}{|}
\newcommand{\VERB}{\Verb[commandchars=\\\{\}]}
\DefineVerbatimEnvironment{Highlighting}{Verbatim}{commandchars=\\\{\}}
% Add ',fontsize=\small' for more characters per line
\usepackage{framed}
\definecolor{shadecolor}{RGB}{248,248,248}
\newenvironment{Shaded}{\begin{snugshade}}{\end{snugshade}}
\newcommand{\AlertTok}[1]{\textcolor[rgb]{0.94,0.16,0.16}{#1}}
\newcommand{\AnnotationTok}[1]{\textcolor[rgb]{0.56,0.35,0.01}{\textbf{\textit{#1}}}}
\newcommand{\AttributeTok}[1]{\textcolor[rgb]{0.13,0.29,0.53}{#1}}
\newcommand{\BaseNTok}[1]{\textcolor[rgb]{0.00,0.00,0.81}{#1}}
\newcommand{\BuiltInTok}[1]{#1}
\newcommand{\CharTok}[1]{\textcolor[rgb]{0.31,0.60,0.02}{#1}}
\newcommand{\CommentTok}[1]{\textcolor[rgb]{0.56,0.35,0.01}{\textit{#1}}}
\newcommand{\CommentVarTok}[1]{\textcolor[rgb]{0.56,0.35,0.01}{\textbf{\textit{#1}}}}
\newcommand{\ConstantTok}[1]{\textcolor[rgb]{0.56,0.35,0.01}{#1}}
\newcommand{\ControlFlowTok}[1]{\textcolor[rgb]{0.13,0.29,0.53}{\textbf{#1}}}
\newcommand{\DataTypeTok}[1]{\textcolor[rgb]{0.13,0.29,0.53}{#1}}
\newcommand{\DecValTok}[1]{\textcolor[rgb]{0.00,0.00,0.81}{#1}}
\newcommand{\DocumentationTok}[1]{\textcolor[rgb]{0.56,0.35,0.01}{\textbf{\textit{#1}}}}
\newcommand{\ErrorTok}[1]{\textcolor[rgb]{0.64,0.00,0.00}{\textbf{#1}}}
\newcommand{\ExtensionTok}[1]{#1}
\newcommand{\FloatTok}[1]{\textcolor[rgb]{0.00,0.00,0.81}{#1}}
\newcommand{\FunctionTok}[1]{\textcolor[rgb]{0.13,0.29,0.53}{\textbf{#1}}}
\newcommand{\ImportTok}[1]{#1}
\newcommand{\InformationTok}[1]{\textcolor[rgb]{0.56,0.35,0.01}{\textbf{\textit{#1}}}}
\newcommand{\KeywordTok}[1]{\textcolor[rgb]{0.13,0.29,0.53}{\textbf{#1}}}
\newcommand{\NormalTok}[1]{#1}
\newcommand{\OperatorTok}[1]{\textcolor[rgb]{0.81,0.36,0.00}{\textbf{#1}}}
\newcommand{\OtherTok}[1]{\textcolor[rgb]{0.56,0.35,0.01}{#1}}
\newcommand{\PreprocessorTok}[1]{\textcolor[rgb]{0.56,0.35,0.01}{\textit{#1}}}
\newcommand{\RegionMarkerTok}[1]{#1}
\newcommand{\SpecialCharTok}[1]{\textcolor[rgb]{0.81,0.36,0.00}{\textbf{#1}}}
\newcommand{\SpecialStringTok}[1]{\textcolor[rgb]{0.31,0.60,0.02}{#1}}
\newcommand{\StringTok}[1]{\textcolor[rgb]{0.31,0.60,0.02}{#1}}
\newcommand{\VariableTok}[1]{\textcolor[rgb]{0.00,0.00,0.00}{#1}}
\newcommand{\VerbatimStringTok}[1]{\textcolor[rgb]{0.31,0.60,0.02}{#1}}
\newcommand{\WarningTok}[1]{\textcolor[rgb]{0.56,0.35,0.01}{\textbf{\textit{#1}}}}
\usepackage{longtable,booktabs,array}
\usepackage{calc} % for calculating minipage widths
% Correct order of tables after \paragraph or \subparagraph
\usepackage{etoolbox}
\makeatletter
\patchcmd\longtable{\par}{\if@noskipsec\mbox{}\fi\par}{}{}
\makeatother
% Allow footnotes in longtable head/foot
\IfFileExists{footnotehyper.sty}{\usepackage{footnotehyper}}{\usepackage{footnote}}
\makesavenoteenv{longtable}
\usepackage{graphicx}
\makeatletter
\def\maxwidth{\ifdim\Gin@nat@width>\linewidth\linewidth\else\Gin@nat@width\fi}
\def\maxheight{\ifdim\Gin@nat@height>\textheight\textheight\else\Gin@nat@height\fi}
\makeatother
% Scale images if necessary, so that they will not overflow the page
% margins by default, and it is still possible to overwrite the defaults
% using explicit options in \includegraphics[width, height, ...]{}
\setkeys{Gin}{width=\maxwidth,height=\maxheight,keepaspectratio}
% Set default figure placement to htbp
\makeatletter
\def\fps@figure{htbp}
\makeatother
\setlength{\emergencystretch}{3em} % prevent overfull lines
\providecommand{\tightlist}{%
  \setlength{\itemsep}{0pt}\setlength{\parskip}{0pt}}
\setcounter{secnumdepth}{-\maxdimen} % remove section numbering
\ifLuaTeX
  \usepackage{selnolig}  % disable illegal ligatures
\fi
\IfFileExists{bookmark.sty}{\usepackage{bookmark}}{\usepackage{hyperref}}
\IfFileExists{xurl.sty}{\usepackage{xurl}}{} % add URL line breaks if available
\urlstyle{same}
\hypersetup{
  pdftitle={February Survey Report},
  pdfauthor={Maksim Zubok},
  hidelinks,
  pdfcreator={LaTeX via pandoc}}

\title{February Survey Report}
\author{Maksim Zubok}
\date{2024-03-15}

\begin{document}
\maketitle

\hypertarget{intro}{%
\subsection{Intro}\label{intro}}

To create weights, I am working with the 2020
\href{https://github.com/antndlcrx/nonviolent-repression/blob/main/data/surveys/Tom3_tab1_VPN-2020.xlsx}{census
data}, particulary the cross tabbed gender, age, and university
education file
\href{https://github.com/antndlcrx/nonviolent-repression/blob/main/data/surveys/ru_population_frame.csv}{here}.

I did not include information about region of residence, even though we
could do it after harmonising census data with the survey.

\begin{Shaded}
\begin{Highlighting}[]
\FunctionTok{head}\NormalTok{(ru\_population\_frame, }\DecValTok{5}\NormalTok{)}
\end{Highlighting}
\end{Shaded}

\begin{verbatim}
## # A tibble: 5 x 4
##   gender  age_group university_education    Freq
##   <fct>   <fct>     <chr>                  <dbl>
## 1 Мужской 18-19     BA+                        1
## 2 Мужской 18-19     BA-                  1367504
## 3 Мужской 18-19     NA                    247138
## 4 Мужской 20-24     BA+                   463546
## 5 Мужской 20-24     BA-                  2727980
\end{verbatim}

Note that because in the survey we have a handful of people aged 18-19
who reported having BA education and because ``survey'' package does not
permit 0 in the population frame, I put 1 in Freq for 18-19 men and
women BA+ intersection.

\hypertarget{sample-to-population-comparison}{%
\subsection{Sample to Population
Comparison}\label{sample-to-population-comparison}}

\begin{verbatim}
## `summarise()` has grouped output by 'gender', 'age_group'. You can override
## using the `.groups` argument.
## `summarise()` has grouped output by 'gender', 'age_group'. You can override
## using the `.groups` argument.
\end{verbatim}

\includegraphics{survey_feb_report_files/figure-latex/unnamed-chunk-2-1.pdf}

The main disparities between Survey Feb and Population are:

\begin{itemize}
\tightlist
\item
  oversampled young women and undersampled young men, especially age
  30-39 for women with university education.
\item
  the opposite for old people. Oversampled older men, undersampled older
  women without university education.
\end{itemize}

\hypertarget{weights-witth-survey-package}{%
\subsection{Weights witth Survey
package}\label{weights-witth-survey-package}}

To compute post-stratification weights we rely on the
\texttt{postStratify} function from the \texttt{survey} package. The
function adjusts the sampling and replicate weights so that the joint
distribution of a set of post-stratifying variables matches the known
population joint distribution. \textbf{However, the package
documentation does not describe how exactly the adjustment is
implemented.}

\begin{Shaded}
\begin{Highlighting}[]
\DocumentationTok{\#\# survey library \#\#}
\NormalTok{unweighted\_data }\OtherTok{\textless{}{-}} \FunctionTok{svydesign}\NormalTok{(}\AttributeTok{ids =} \SpecialCharTok{\textasciitilde{}}\DecValTok{1}\NormalTok{, }\AttributeTok{data =}\NormalTok{ survey\_feb)}


\NormalTok{weighted }\OtherTok{\textless{}{-}} \FunctionTok{postStratify}\NormalTok{(unweighted\_data, }\SpecialCharTok{\textasciitilde{}}\NormalTok{age\_group }\SpecialCharTok{+}\NormalTok{ gender }\SpecialCharTok{+}\NormalTok{ university\_education,}
\NormalTok{             ru\_population\_frame, }\AttributeTok{partial=}\ConstantTok{TRUE}\NormalTok{)}

\CommentTok{\# save weights }
\NormalTok{survey\_feb}\SpecialCharTok{$}\NormalTok{weight }\OtherTok{\textless{}{-}} \FunctionTok{weights}\NormalTok{(weighted)}
\NormalTok{sum\_feb }\OtherTok{\textless{}{-}} \FunctionTok{summary}\NormalTok{(}\FunctionTok{weights}\NormalTok{(weighted))}
\NormalTok{sum\_feb}
\end{Highlighting}
\end{Shaded}

\begin{verbatim}
##    Min. 1st Qu.  Median    Mean 3rd Qu.    Max. 
##       0   18588   33196   62445   52625 3604783
\end{verbatim}

Note: some strata had no observations in the survey (NA on education for
some age gender groups). This means we had to ignore them in producing
weights.

Also the weights for women with no uni education aged 70+ seem to be too
large. It is not surprising given the disparity, but noteworthy.

\begin{Shaded}
\begin{Highlighting}[]
\DocumentationTok{\#\# survey library \#\#}
\NormalTok{unweighted\_data }\OtherTok{\textless{}{-}} \FunctionTok{svydesign}\NormalTok{(}\AttributeTok{ids =} \SpecialCharTok{\textasciitilde{}}\DecValTok{1}\NormalTok{, }\AttributeTok{data =}\NormalTok{ survey\_aug)}


\NormalTok{weighted }\OtherTok{\textless{}{-}} \FunctionTok{postStratify}\NormalTok{(unweighted\_data, }\SpecialCharTok{\textasciitilde{}}\NormalTok{age\_group }\SpecialCharTok{+}\NormalTok{ gender }\SpecialCharTok{+}\NormalTok{ university\_education,}
\NormalTok{             ru\_population\_frame, }\AttributeTok{partial=}\ConstantTok{TRUE}\NormalTok{)}

\CommentTok{\# save weights }
\NormalTok{survey\_aug}\SpecialCharTok{$}\NormalTok{weight }\OtherTok{\textless{}{-}} \FunctionTok{weights}\NormalTok{(weighted)}
\NormalTok{sum\_aug }\OtherTok{\textless{}{-}} \FunctionTok{summary}\NormalTok{(}\FunctionTok{weights}\NormalTok{(weighted))}
\NormalTok{sum\_aug}
\end{Highlighting}
\end{Shaded}

\begin{verbatim}
##    Min. 1st Qu.  Median    Mean 3rd Qu.    Max. 
##   15425   25753   43376   65425   77942 1216293
\end{verbatim}

For August, we also see that some weights are much larger than others.
As you can see in the graphs below, the distribution of weights is
similarly skewed and the disparities between the bulk of the
distribution and its tales are in the same orders of magnitude. However,
the largest weight in Feb survey is three times bigger than the largest
weight in Aug survey.

The largest weights in both surveys relate to different population
groups.

\begin{longtable}[]{@{}cccc@{}}
\caption{February Survey, Top Five Rows by Weight}\tabularnewline
\toprule\noalign{}
age\_group & gender & university\_education & weight \\
\midrule\noalign{}
\endfirsthead
\toprule\noalign{}
age\_group & gender & university\_education & weight \\
\midrule\noalign{}
\endhead
\bottomrule\noalign{}
\endlastfoot
70+ & Женский & BA- & 3604783.0 \\
70+ & Женский & BA- & 3604783.0 \\
35-39 & Женский & NA & 1302931.0 \\
60-64 & Женский & BA- & 811229.8 \\
60-64 & Женский & BA- & 811229.8 \\
\end{longtable}

\begin{longtable}[]{@{}cccc@{}}
\caption{August Survey, Top Five Rows by Weight}\tabularnewline
\toprule\noalign{}
age\_group & gender & university\_education & weight \\
\midrule\noalign{}
\endfirsthead
\toprule\noalign{}
age\_group & gender & university\_education & weight \\
\midrule\noalign{}
\endhead
\bottomrule\noalign{}
\endlastfoot
30-34 & Женский & NA & 1216293 \\
40-44 & Мужской & NA & 1052747 \\
60-64 & Женский & NA & 856270 \\
70+ & Мужской & BA- & 797236 \\
70+ & Мужской & BA- & 797236 \\
\end{longtable}

\includegraphics{survey_feb_report_files/figure-latex/unnamed-chunk-7-1.pdf}

\includegraphics{survey_feb_report_files/figure-latex/unnamed-chunk-8-1.pdf}

\hypertarget{weights-created-manually}{%
\subsection{Weights created manually}\label{weights-created-manually}}

To check the plausibility of resulting weights, we create alternative
weights based on the population frequencies of the combination of the
same strata (Yana's approach). The weights are calculated for each
category:
\[ \text{weight}_i = \frac{\text{population frequency}_i}{\text{sample frequency}_i} \]

\begin{Shaded}
\begin{Highlighting}[]
\CommentTok{\# calculate weights and print the df with weights}
\NormalTok{weights\_aug\_strata }\OtherTok{\textless{}{-}} \FunctionTok{left\_join}\NormalTok{(survey\_aug\_strata,}
\NormalTok{                            pop\_strata,}
                            \FunctionTok{c}\NormalTok{(}\StringTok{"gender"}\NormalTok{, }\StringTok{"age\_group"}\NormalTok{, }\StringTok{"university\_education"}\NormalTok{)) }\SpecialCharTok{\%\textgreater{}\%} 
  \FunctionTok{rename}\NormalTok{(}\AttributeTok{population\_proportion =}\NormalTok{ proportion.y,}
         \AttributeTok{sample\_proportion =}\NormalTok{ proportion.x)  }\SpecialCharTok{\%\textgreater{}\%} 
  \CommentTok{\# calculate weights as popul prop/sample prop}
  \FunctionTok{mutate}\NormalTok{(}\AttributeTok{weight =}\NormalTok{ population\_proportion }\SpecialCharTok{/}\NormalTok{ sample\_proportion)}

\NormalTok{sum\_man\_aug }\OtherTok{\textless{}{-}}\FunctionTok{summary}\NormalTok{(weights\_aug\_strata}\SpecialCharTok{$}\NormalTok{weight)}
\NormalTok{sum\_man\_aug}
\end{Highlighting}
\end{Shaded}

\begin{verbatim}
##    Min. 1st Qu.  Median    Mean 3rd Qu.    Max. 
##  0.2105  0.5255  1.0878  2.4029  1.9037 16.5951
\end{verbatim}

\begin{Shaded}
\begin{Highlighting}[]
\CommentTok{\# calculate weights and print the df with weights}
\NormalTok{weights\_feb\_strata }\OtherTok{\textless{}{-}} \FunctionTok{left\_join}\NormalTok{(survey\_feb\_strata,}
\NormalTok{                            pop\_strata,}
                            \FunctionTok{c}\NormalTok{(}\StringTok{"gender"}\NormalTok{, }\StringTok{"age\_group"}\NormalTok{, }\StringTok{"university\_education"}\NormalTok{)) }\SpecialCharTok{\%\textgreater{}\%} 
  \FunctionTok{rename}\NormalTok{(}\AttributeTok{population\_proportion =}\NormalTok{ proportion.y,}
         \AttributeTok{sample\_proportion =}\NormalTok{ proportion.x)  }\SpecialCharTok{\%\textgreater{}\%} 
  \CommentTok{\# calculate weights as popul prop/sample prop}
  \FunctionTok{mutate}\NormalTok{(}\AttributeTok{weight =}\NormalTok{ population\_proportion }\SpecialCharTok{/}\NormalTok{ sample\_proportion)}

\CommentTok{\# head(weights\_feb\_strata)}
\NormalTok{sum\_man\_feb }\OtherTok{\textless{}{-}} \FunctionTok{summary}\NormalTok{(weights\_feb\_strata}\SpecialCharTok{$}\NormalTok{weight)}
\NormalTok{sum\_man\_feb}
\end{Highlighting}
\end{Shaded}

\begin{verbatim}
##    Min. 1st Qu.  Median    Mean 3rd Qu.    Max. 
##  0.0000  0.5014  1.1922  3.3786  3.0920 49.1836
\end{verbatim}

\begin{verbatim}
## Adding missing grouping variables: `gender`, `age_group`
## Adding missing grouping variables: `gender`, `age_group`
\end{verbatim}

\includegraphics{survey_feb_report_files/figure-latex/unnamed-chunk-12-1.pdf}

\includegraphics{survey_feb_report_files/figure-latex/unnamed-chunk-13-1.pdf}

Calculating weights this way gives us the same overall picture. The main
difference is in the orders of magnitude between the median weight and
the tail (highest weight).

For \textbf{February} survey, the difference (max weight / median
weight) with \texttt{postStratify} is \textbf{108.59}, but with simple
manually created weights it is \textbf{41.26}. For \textbf{August}
survey, the difference with \texttt{postStratify} is \textbf{28.04}, but
with simple manually created weights it is \textbf{15.26}.

Note that some observations in February survey have a weight of zero
because they do not exist in the population census. These are women
18-19 years old with university education. My suspicion on why we have
these observations is that some respondents might have misreported their
education attainment.

\end{document}
